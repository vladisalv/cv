%%%%%%%%%%%%%%%%%%%%%%%%%%%%%%%%%%%%%%%%%%%%%%%%%%%%%%%%%%%%%%%%%%%%%%%%%%%%%%%%
%
% This is Curriculum Vitae (resume) of Vladislav Toigildin.
%
% LICENSE: This source file is subject to MIT license.
%
% IMPORTANT: THIS FILE NEEDS TO BE COMPILED WITH XeLaTeX
%
% I used Constantia font and classic blue theme.
% Also, i liked Garamond, Georgia and Helvetica fonts.
%
% For compile 4 different documents run `make` without any arguments.
%
% Check fonts embedded to pdf before publish your cv (use pdffonts).
%
% Project address: https://github.com/vladisalv/cv
%
%%%%%%%%%%%%%%%%%%%%%%%%%%%%%%%%%%%%%%%%%%%%%%%%%%%%%%%%%%%%%%%%%%%%%%%%%%%%%%%%

\documentclass[12pt,a4paper,sans]{moderncv} % http://www.ctan.org/pkg/moderncv
\usepackage[scale=0.75]{geometry}
\usepackage{fancyhdr} % for footer
\usepackage{changepage} % for adjustwidth
\graphicspath{{images/}} % path to images
\hyphenpenalty=10000 % ban of hyphenation

%-------------------------------------------------------------------------------
% CREATE AND SET NEW TOGGLE
%-------------------------------------------------------------------------------

\newtoggle{langEnglish}
\newtoggle{langRussian}

\toggletrue{langEnglish}
\togglefalse{langRussian}

\newtoggle{compact}
\newtoggle{detailed}

\toggletrue{compact}
\togglefalse{detailed}

%-------------------------------------------------------------------------------
% NEW COMMAND
%-------------------------------------------------------------------------------

\newcommand\ifjobname[2]{
  \expandafter\ifstrequal\expandafter{\jobname}{TOIGILDIN_VLADISLAV_dev#1}
  {#2}
  {}
}

\newcommand\EngRus[2]{\iftoggle{langEnglish}{#1}{\iftoggle{langRussian}{#2}{}}}

\newcommand\ifdetailed[1]{\iftoggle{detailed}{#1}{}}

\newcommand\ifcompact[1]{\iftoggle{compact}{#1}{}}

\newcommand\iflangEnglish[1]{\iftoggle{langEnglish}{#1}{}}

\newcommand\iflangRussian[1]{\iftoggle{langRussian}{#1}{}}

%-------------------------------------------------------------------------------
% PARSE VERSION OF CV FROM \jobname
%-------------------------------------------------------------------------------

\moderncvtheme[blue]{classic} % theme by default
\def\myfont{Constantia}       % font by default
\ifjobname{_diff}{            % take theme from file style.tex
  \input{style.tex}
}

\ifjobname{_eng_compact}{
  \toggletrue{langEnglish}
  \togglefalse{langRussian}
  \toggletrue{compact}
  \togglefalse{detailed}
}

\ifjobname{_rus_compact}{
  \togglefalse{langEnglish}
  \toggletrue{langRussian}
  \toggletrue{compact}
  \togglefalse{detailed}
}

\ifjobname{_eng_detailed}{
  \toggletrue{langEnglish}
  \togglefalse{langRussian}
  \togglefalse{compact}
  \toggletrue{detailed}
}

\ifjobname{_rus_detailed}{
  \togglefalse{langEnglish}
  \toggletrue{langRussian}
  \togglefalse{compact}
  \toggletrue{detailed}
}

%-------------------------------------------------------------------------------
% SET OPTION PACKAGE
%-------------------------------------------------------------------------------

\usepackage{fontspec}
\usepackage{polyglossia} % load multi language
\iflangEnglish{
  \setdefaultlanguage{english}
  \setmainfont{\myfont}
}
\iflangRussian{
  \setdefaultlanguage[spelling=modern]{russian}
  \setotherlanguage{english} % second language
  \defaultfontfeatures{Ligatures={TeX},Renderer=Basic}
  \setmainfont{\myfont}
  \newfontfamily\cyrillicfont{\myfont}
  \newfontfamily\cyrillicfontsf{\myfont}
  \newfontfamily\cyrillicfonttt{\myfont}
}

\ifcompact{
  \nopagenumbers{}
}
\ifdetailed{
  \usepackage{lastpage}
  \pagestyle{fancy}
  \rfoot{\thepage/\pageref{LastPage}}
}

%-------------------------------------------------------------------------------
% PERSONAL INFORMATION
%-------------------------------------------------------------------------------

\firstname{\EngRus{Vladislav}{Владислав}}
\familyname{\EngRus{Toigildin}{Тойгильдин}}
\title{Software Developer}
\address{\EngRus{Moscow, Russia}{Россия, Москва}}
\phone[mobile]{+7~916~576~66~39}
\email{vtoigildin@protonmail.com}
\extrainfo{%
  \\%
  \includegraphics[height=14pt]{github_logo.png}~\httplink[vladisalv]{github.com/vladisalv}\\
  \includegraphics[height=14pt]{linkedin_logo.png}~\httplink[vladisalv]{linkedin.com/in/vladisalv}
}
\photo[64pt][0.4pt]{photo.jpg}

%-------------------------------------------------------------------------------
% BEGIN DOCUMENT
%-------------------------------------------------------------------------------

\begin{document}

\makecvtitle

% DEBUG OUTPUT
\newtoggle{DEBUG}
%\toggletrue{DEBUG}
\togglefalse{DEBUG}

\iftoggle{DEBUG}{
  \iftoggle{langEnglish}{You use English}{}
  \iftoggle{langRussian}{You use Russian}{}
  \iftoggle{compact}{You use compact}{}
  \iftoggle{detailed}{You use detailed}{}
  \jobname
}

%\ifdetailed{
%  \section{\EngRus{Objective}{Цель}}
%  \begin{adjustwidth}{2.5cm}{0cm}
%    \begin{center}
%      \EngRus{Plan}
%             {Планирую}
%    \end{center}
%  \end{adjustwidth}
%}

%-------------------------------------------------------------------------------
% EXPERIENCE
%-------------------------------------------------------------------------------

\section{\EngRus{Experience}{Опыт работы}}

\cventry
  {11.2016 -- 01.2019}
  {\EngRus{Deep Learning Performance Engineer}{Инженер-программист}}
  {NVIDIA}
  {\EngRus{Moscow}{Москва}}
  {}
  {
    \ifcompact{
      \EngRus{{\itshape Development of a system for benchmarking DL frameworks (TensorFlow, PyTorch and etc).}}
             {{\itshape Разработка системы для тестирования DL фреймворков (TensorFlow, PyTorch and etc).}}
    }
    \ifdetailed{
      \EngRus{{Development of a system for benchmarking DL (TensorFlow, PyTorch and etc) frameworks using GPUs.}}
             {{Разработка системы для тестирования производительности DL фреймворков (TensorFlow, PyTorch and etc) на графических ускорителях.}}
      \vfill
      \begin{adjustwidth}{0.5cm}{0cm}
        \begin{itemize}
          \item \EngRus{Developed a toolchain for automation of the test process.\newline
                        Speed it up in 5 times.}
                       {Разработал набор инструментов для автоматизации тестирования.
                        Ускорил весь цикл проведения тестов в 5 раз.}
          \item \EngRus{I was a senior maintainer
                        \href{https://developer.nvidia.com/deep-learning-performance-training-inference}{\textcolor{blue}{\underline{for monthly baselines}}}
                        performance data for a year.}
                       {Больше года был основным мейнтейнером данных
                        \href{https://developer.nvidia.com/deep-learning-performance-training-inference}{\textcolor{blue}{\underline{для~ежемесячных~отчетов}}}
                        о~производительности нашего~железа.}
          \item \EngRus{Developed DL benchmarks: preparing data and models, deploying, running,
                        collecting data, uploading stats into database.}
                       {Разрабатывал бенчмарки для DL фреймворков:
                        подготавка данных и моделей, деплой на сервера,
                        запуск, сбор статистики, выгрузка результатов в базы данных.}
          \item \EngRus{Updated legacy Perl code to Python.}
                       {Обновлял legacy-код приложений c Perl на Python.}
          \item \EngRus{Participated in the design process of cloud container infrastructure.}
                       {Принимал участие в создании облачной инфраструктуры.}
        \end{itemize}
      \end{adjustwidth}
    }
  }

\cventry
  {09.2014 -- 08.2016}
  {\EngRus{Researcher}{Исследователь}}
  {\EngRus{Research Computing Center MSU}
          {Научно-исследовательский вычислительный центр МГУ}}
  {\EngRus{Moscow}{Москва}}
  {}
  {
    \ifcompact{
      \EngRus{{\itshape Design and development of a parallel version of the algorithm
              of repeats search in biological sequence.}}
             {{\itshape Проектирование и разработка параллельной версии алгоритма
              поиска повторов в биологических последовательностях.}}
    }
    \ifdetailed{
      \EngRus{Design and development of a parallel version of algorithm
              for genome blurred repeats search.}
             {Проектирование и разработка параллельной версии алгоритма
              поиска неточных повторов в геноме.}
      \vfill
      \begin{adjustwidth}{0.5cm}{0cm}
        \begin{itemize}
          \item \EngRus{Developed a parallel spectral-analytical method
                        for heterogeneous distributed multiprocessing systems.}
                       {Разработал алгоритм для параллельного спектрально-аналитического метода
                        для гетерогенных многопроцессорных систем с распределенной памятью.}
          \item \EngRus{Designed a object-oriented architecture for flexible balancing between computational nodes.}
                       {Спроектировал объектно-ориентированную архитектуру, позволяющую
                        настраивать гибкую балансировку между узлами вычислительной сети.}
          \item \EngRus{Developed the parallel program using MPI and CUDA.}
                       {Разработал параллельную программу, использующую технологию
                        обмена сообщениями и графические ускорители (C++/MPI/Cuda).}
          \item \EngRus{Optimized app for linear scale up to 2048 processors.}
                       {Поднял масштабируемость алгоритма до линейной вплоть до 2048 процессоров.}
        \end{itemize}
      \end{adjustwidth}
    }
  }

\cventry
  {06.2015 -- 03.2016}
  {\EngRus{Software engineer}{Инженер-программист}}
  {IBM}
  {\EngRus{Moscow}{Москва}}
  {}
  {
    \ifcompact{
      \EngRus{{\itshape Linux on IBM z System SAN component development.}}
             {{\itshape Разработка подсистемы SAN ОС Linux для аппаратной платформы IBM z System.}}
    }
    \ifdetailed{
      \EngRus{{Development of Linux driver (zfcp) for IBM z System (s390x) storage hardware.}}
             {{Разработка Linux драйвера для устройств хранения данных
              аппаратной платформы IBM z System (s390x).}}
      \vfill
      \begin{adjustwidth}{0.5cm}{0cm}
        \begin{itemize}
          \item \EngRus{Developed Linux driver for SCSI devices.}
                       {Разрабатывал Linux драйвер для SCSI-устройств хранения данных.}
          \item \EngRus{Modified internal disk perf analyze tool (C++ and Perl).}
                       {Модифицировал внутренюю утилиту анализа производительности дисков (C++ и Perl).}
          \item \EngRus{Designed and implemented stress testing system.}
                       {Спроектировал и разработал систему стресс-тестирования.}
        \end{itemize}
      \end{adjustwidth}
    }
  }

\cventry
  {11.2013 -- 10.2014}
  {\EngRus{Technician (Part Time)}{Техник-программист (частичная занятость)}}
  {\EngRus{Nuclear Safety Institute of~the~Russian Academy of~Sciences}{ИБРАЭ РАН}}
  {\EngRus{Moscow}{Москва}}
  {}
  {
    \EngRus{{\ifcompact{\itshape}Development of a model of hydrodynamic process in liquids using CABARET scheme.}}
           {{\ifcompact{\itshape}Разработка модели течения вязких жидкостей использующей схему КАБАРЕ.}}
    \ifdetailed{
      \vfill
      \begin{adjustwidth}{0.5cm}{0cm}
        \begin{itemize}
          \item \EngRus{Designed and implemented GUI (Qt).}
                       {Спроектировал и реализовал графический интерфейс основной программы (Qt).}
          \item \EngRus{Added GPU computing support (Cuda)}
                       {Добавил поддержку вычислений на графических устройствах (Cuda).}
        \end{itemize}
      \end{adjustwidth}
    }
  }

%-------------------------------------------------------------------------------
% TECHNICAL SKILLS
%-------------------------------------------------------------------------------

\section{\EngRus{Technical skills}{Технические навыки}}

\cvline
  {Languages}
  {Python, Perl, Bash, C, C++}
\cvline
  {Frameworks}
  {Flask, Django}
\cvline
  {OS}
  {Linux}
\cvline
  {CI/CD}
  {Docker, GitLab CI}
\cvline
  {SQL}
  {PostgreSQL}
\ifdetailed{
  \cvline
    {VCS}
    {Git}
  \cvline
    {HPC}
    {MPI, Cuda, OpenMP}
  \cvline
    {Builder}
    {Make, Autotools}
  \cvline
    {Others}
    {Qt, \LaTeX, Gnu plot}
}

%-------------------------------------------------------------------------------
% EDUCATION
%-------------------------------------------------------------------------------

\section{\EngRus{Education}{Образование}}

\cventry
  {2010 -- 2015}
  {\EngRus{MSc (equivalent) in Applied Mathematics and Computer Science}
          {Специалист по прикладной математике и информатике}%
  }
  {\EngRus{Lomonosov Moscow State University}{\newline МГУ им. М.В. Ломоносова}}
  {\EngRus{Moscow}{Москва}}
  {\ifdetailed{%
     \EngRus{Faculty of Computational Mathematics and Cybernetics}
            {Факультет Вычислительной Математики и Кибернетики}%
   }%
  }
  {\ifdetailed{
     \vfill
     \begin{adjustwidth}{0.5cm}{0cm}
       \begin{itemize}
         \item \EngRus{Qualification: specialist in mathematics and system programming}
                      {Квалификация: математик, системный программист}
         \item \EngRus{Department of Supercomputers and Quantum Informatics}
                      {Кафедра Суперкомпьютеров и Квантовой Информатики}
         \item \EngRus{Specialization: high performance computing}
                      {Специализация: высокопроизводительные вычисления}
         \item \EngRus{Master thesis "Research and development of parallel
                       algorithm for genome blurred repeats search"}
                      {Дипломная работа "Разработка и исследование параллельного
                       алгоритма поиска неточных повторов в геноме"}
         \item \EngRus{Knowledges: Algorithms and Data structures, Parallel data processing,
                       Operating systems, Databases, Computer architecture and assembler language,
                       Mathematical analysis, Discrete mathematics, Numerical methods and others.}
                      {Прослушанные курсы: Алгоритмы и алгоритмические языки,
                       Суперкомпьютеры и параллельная обработка данных,
                       Операционные системы, Базы данных, Архитектура ЭВМ и язык ассемблера,
                       Математический анализ, Дискретная математика, Численные методы и другие.}
       \end{itemize}
     \end{adjustwidth}
   }
  }

%-------------------------------------------------------------------------------
% PUBLICATIONS, AWARDS, OPEN SOURCE PROJECT, ADDITIONAL INFO AND FOOTER
%-------------------------------------------------------------------------------

% ---------- Publications ------------------------------------------------------
\ifdetailed{
  \renewcommand\refname{\EngRus{Publications}{Публикации}}
  \begin{thebibliography}{0}

  \bibitem{ISP}
    \EngRus{{\itshape A.N. Pankratov, R.K. Tetuev, M.I. Pyatkov, V.P. Toigildin, N.N. Popova}
            Spectral analytical method of recognition of inexact repeats in character sequences.~--~
            Proceedings of the Institute for System Programming Volume 27 (Issue 6). 2015 y. pp. 335-344.
            \href{http://www.ispras.ru/en/proceedings/isp_27_2015_6/isp_27_2015_6_335/}
                 {\textcolor{blue}{\underline{Abstract}}}}
           {{\itshape Панкратов А.Н., Тетуев Р.К., Пятков М.И., Тойгильдин В.П., Попова Н.Н.}
            Спектрально-аналитический метод распознавания неточных повторов в символьных последовательностях.~--~
            Труды Института системного программирования РАН Том 27. Выпуск 6. 2015 г.
            \href{http://www.ispras.ru/proceedings/docs/2015/27/6/isp_27_2015_6_335.pdf}
                 {\textcolor{blue}{\underline{Стр. 335-344}}}}

    \bibitem{cuda}
      \EngRus{{\itshape V.P. Toigildin} Research and development of parallel algorithm
              for~genome blurred repeats search. -- CUDA Almanac, 2015 February.~--~
              \href{http://www.nvidia.ru/docs/IO/141194/CUDA-\%D0\%B0\%D0\%BB\%D1\%8C\%D0\%BC\%D0\%B0\%D0\%BD\%D0\%B0\%D1\%85-feb-2015.pdf}
                   {\textcolor{blue}{\underline{p.12}}}}
             {{\itshape Тойгильдин В.П.} Разработка и исследование параллельного алгоритма
              поиска неточных повторов в геноме. -- CUDA Альманах, Февраль 2015.~--~
              \href{http://www.nvidia.ru/docs/IO/141194/CUDA-\%D0\%B0\%D0\%BB\%D1\%8C\%D0\%BC\%D0\%B0\%D0\%BD\%D0\%B0\%D1\%85-feb-2015.pdf}
                   {\textcolor{blue}{\underline{Стр. 12}}}}

  \end{thebibliography}
}
% ---------- End Publications --------------------------------------------------

% ---------- Awards ------------------------------------------------------------
\ifdetailed{
  \section{\EngRus{Awards}{Награды}}
  \cventry
    {2014}
    {\EngRus{CUDA Center of Excellence MSU Grant}
            {Стипендия от CUDA Center of Excellence МГУ}%
    }
    {}
    {\EngRus{Moscow}{Москва}}
    {}
    {\EngRus{\normalsize \href{http://ccoe.msu.ru/ru/node/64}{\textcolor{blue}{\underline{Won a grant}}}
             for significant acceleration of computing for~my~research by using GPU.}
            {\normalsize \href{http://ccoe.msu.ru/ru/node/64}{\textcolor{blue}{\underline{Выиграл стипендию}}}
            за существенное ускорение вычислений для своей исследовательской работы
            за счет применения графических процессоров.}%
    }
}
% ---------- End Awards --------------------------------------------------------

% ---------- Open Source -------------------------------------------------------
\ifdetailed{
  \section{\EngRus{Open Source Project}{Open Source Проекты}}
  \cvline
    {\href{https://github.com/vladisalv/mpisbars}{\textcolor{blue}{\underline{mpiSBARS}}}}
    {\EngRus{Parallel program for recognition of extended inexact repeats in~genome.
             MPI+CUDA model is used for better scalability on heterogeneous high performance systems.}
            {Параллельная программа для поиска неточных протяженных повторов
             в биологических последовательностях. Используется модель MPI+CUDA
             для достижения лучшей масштабируемости на современных гетерогенных системах.}%
    }
}
% ---------- End Open Source ---------------------------------------------------

% ---------- Additional information --------------------------------------------
\section{\EngRus{Additional information}{Дополнительная информация}}
\cvline
  {\EngRus{Languages}{Языки}}
  {\EngRus{English(intermediate), Russian(native)}
          {Английский(средний уровень), Русский(родной)}}
\cvline
  {\EngRus{Interests}{Интересы}}
  {\EngRus{Improv theatre}{Сценическая импровизация}}
% ---------- Additional information --------------------------------------------

% ---------- Footer ------------------------------------------------------------
\fancyfoot[C]{\scriptsize ~~~~~~~~~~~~~~~~~~~~~~~~~~~
              You can find recent and more detailed version cv
              \href{https://github.com/vladisalv/cv}{\underline{here}}.\newline
              Last updated: \today\-}
% ---------- End Footer --------------------------------------------------------

% @PLAN: English(IELTS 7.0 Overall: 6.0L, 7.0R, 8.0W, 7.0S) after IELTS
% @PLAN: add section Communication Skills/Public Activity
% @PLAN: section Education:
% \subsection{University}
% \subsection{MOOOCs (Coursera)} (https://en.wikipedia.org/wiki/Massive_open_online_course)

\end{document}
