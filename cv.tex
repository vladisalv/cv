%%%%%%%%%%%%%%%%%%%%%%%%%%%%%%%%%%%%%%%%%%%%%%%%%%%%%%%%%%%%%%%%%%%%%%%%%%%%%%%%
%
% This is Curriculum Vitae (resume) of Vladislav Toigildin.
%
% LICENSE: This source file is subject to MIT license.
%
% IMPORTANT: THIS FILE NEEDS TO BE COMPILED WITH XeLaTeX
%
% Project address: https://github.com/vladisalv/cv
%
%%%%%%%%%%%%%%%%%%%%%%%%%%%%%%%%%%%%%%%%%%%%%%%%%%%%%%%%%%%%%%%%%%%%%%%%%%%%%%%%

\documentclass[12pt,a4paper,sans]{moderncv} % http://www.ctan.org/pkg/moderncv
\usepackage[scale=0.75]{geometry}
\usepackage{fancyhdr} % for footer
\usepackage{changepage} % for adjustwidth
\graphicspath{{images/}} % path to images
\hyphenpenalty=10000 % ban of hyphenation

%-------------------------------------------------------------------------------
% CREATE AND SET NEW TOGGLE
%-------------------------------------------------------------------------------

\newtoggle{langEnglish}
\newtoggle{langRussian}

\toggletrue{langEnglish}
\togglefalse{langRussian}

\newtoggle{compact}
\newtoggle{detailed}

\toggletrue{compact}
\togglefalse{detailed}

%-------------------------------------------------------------------------------
% NEW COMMAND
%-------------------------------------------------------------------------------

\newcommand\ifjobname[2]{
  \expandafter\ifstrequal\expandafter{\jobname}{TOIGILDIN_VLADISLAV_dev#1}
  {#2}
  {}
}

\newcommand\EngRus[2]{\iftoggle{langEnglish}{#1}{\iftoggle{langRussian}{#2}{}}}

\newcommand\ifdetailed[1]{\iftoggle{detailed}{#1}{}}

\newcommand\ifcompact[1]{\iftoggle{compact}{#1}{}}

\newcommand\iflangEnglish[1]{\iftoggle{langEnglish}{#1}{}}

\newcommand\iflangRussian[1]{\iftoggle{langRussian}{#1}{}}

%-------------------------------------------------------------------------------
% PARSE VERSION OF CV FROM \jobname
%-------------------------------------------------------------------------------

\moderncvtheme[blue]{classic} % theme by default
\def\myfont{Constantia}       % font by default
\ifjobname{_diff}{            % take theme from file style.tex
  \input{style.tex}
}

\ifjobname{_eng_compact}{
  \toggletrue{langEnglish}
  \togglefalse{langRussian}
  \toggletrue{compact}
  \togglefalse{detailed}
}

\ifjobname{_rus_compact}{
  \togglefalse{langEnglish}
  \toggletrue{langRussian}
  \toggletrue{compact}
  \togglefalse{detailed}
}

\ifjobname{_eng_detailed}{
  \toggletrue{langEnglish}
  \togglefalse{langRussian}
  \togglefalse{compact}
  \toggletrue{detailed}
}

\ifjobname{_rus_detailed}{
  \togglefalse{langEnglish}
  \toggletrue{langRussian}
  \togglefalse{compact}
  \toggletrue{detailed}
}

%-------------------------------------------------------------------------------
% SET OPTION PACKAGE
%-------------------------------------------------------------------------------

\usepackage{fontspec}
\usepackage{polyglossia} % load multi language
\iflangEnglish{
  \setdefaultlanguage{english}
  \setmainfont{\myfont}
}
\iflangRussian{
  \setdefaultlanguage[spelling=modern]{russian}
  \setotherlanguage{english} % second language
  \defaultfontfeatures{Ligatures={TeX},Renderer=Basic}
  \setmainfont{\myfont}
  \newfontfamily\cyrillicfont{\myfont}
  \newfontfamily\cyrillicfontsf{\myfont}
  \newfontfamily\cyrillicfonttt{\myfont}
}

\ifcompact{
  \nopagenumbers{}
}
\ifdetailed{
  \usepackage{lastpage}
  \pagestyle{fancy}
  \rfoot{\thepage/\pageref{LastPage}}
}

%-------------------------------------------------------------------------------
% PERSONAL INFORMATION
%-------------------------------------------------------------------------------

\firstname{\EngRus{Vladislav}{Владислав}}
\familyname{\EngRus{Toigildin}{Тойгильдин}}
\title{Software Development Engineer}
\address{\EngRus{Moscow, Russia}{Россия, Москва}}
\phone[mobile]{+7~916~576~66~39}
\email{vladisalv@yandex.ru}
%\extrainfo{%
%  \\%
%  \includegraphics[height=14pt]{github_logo.png}~\httplink[vladisalv]{github.com/vladisalv}\\
%% \includegraphics[height=14pt]{linkedin_logo.png}~\httplink[vladisalv]{linkedin.com/in/vladisalv}
%}
%\photo[64pt][0.4pt]{photo.jpg}


%\definecolor{dark-gray}{gray}{0.2}
%\quote{{\raggedleft\color{dark-gray}
%  \EngRus{Looking for a backend developer position in a product team.}
%         {Ищу позицию backend-разработчика в~продуктовой~команде.}
%}}
%\let\originalrecomputecvlengths\recomputecvlengths
%\renewcommand*{\recomputecvlengths}{%
%\originalrecomputecvlengths%
%\setlength{\quotewidth}{0.8\textwidth}}

%-------------------------------------------------------------------------------
% BEGIN DOCUMENT
%-------------------------------------------------------------------------------

\begin{document}

\makecvtitle
\vspace*{-1cm}

% DEBUG OUTPUT
\newtoggle{DEBUG}
%\toggletrue{DEBUG}
\togglefalse{DEBUG}

\iftoggle{DEBUG}{
  \iftoggle{langEnglish}{You use English}{}
  \iftoggle{langRussian}{You use Russian}{}
  \iftoggle{compact}{You use compact}{}
  \iftoggle{detailed}{You use detailed}{}
  \jobname
}

%-------------------------------------------------------------------------------
% EXPERIENCE
%-------------------------------------------------------------------------------

\section{\EngRus{Experience}{Опыт работы}}

\cventry
  {01.2023 -- 09.2023}
  {\EngRus{Software Developer}{Разработчик}}
  {Avito}
  {\EngRus{Moscow}{Москва}}
  {}
  {
      \EngRus{Goods b2b department}{Goods b2b department}
      \begin{adjustwidth}{0.5cm}{0cm}
        \begin{itemize}
          \item \EngRus{Leaded as an architect a new internal service.}{}
          \item \EngRus{Developed and advocated solutions at architectural committees.}{}
          \item \EngRus{Released a new service to production for 5 months with 4-people team.}{}
          \item \EngRus{Reduced a t2m business process (category restructing)\newline from 2 months to 1 week.}{}
        \end{itemize}
      \end{adjustwidth}
  }

\cventry
  {03.2021 -- 03.2022}
  {\EngRus{Software Developer}{Разработчик}}
  {Tinkoff}
  {\EngRus{Moscow}{Москва}}
  {}
  {
    \EngRus{ML Core Platform}{ML Core Platform}
    \begin{adjustwidth}{0.5cm}{0cm}
      \begin{itemize}
        \item \EngRus{Designed and developed a microservice architecture since first release.}
                     {Разрабатывал микросервисную архитектуру системы с момента запуска платформы.}
        \item \EngRus{Fully responsible for a client application.}
                     {Полностью отвечал за клиентское приложение.}
      \end{itemize}
    \end{adjustwidth}
  }

\cventry
  {11.2016 -- 01.2019}
  {\EngRus{Deep Learning Performance Engineer}{Инженер-программист}}
  {NVIDIA}
  {\EngRus{Moscow}{Москва}}
  {}
  {
    \EngRus{Perf Lab}{Perf Lab}
    \begin{adjustwidth}{0.5cm}{0cm}
      \begin{itemize}
        \item \EngRus{Redesigned a Deep Learning benchmark system that speeded a monthly test workflow up 5 times and increased reliability.}{}
        \item \EngRus{I was a senior maintainer
                       \href{https://developer.nvidia.com/deep-learning-performance-training-inference}
                       {\textcolor{blue}{\underline{for monthly baselines}}}
                       performance data.\newline200k+ perf tests at the end, personal responsibility.}
                     {}
        \item \EngRus{Modified Perl codebase to Python, that improved maintainability.}{}
        \item \EngRus{Actively troubleshot issues on Linux production servers.}{}
      \end{itemize}
    \end{adjustwidth}
  }

\cventry
  {06.2015 -- 03.2016}
  {\EngRus{Software engineer}{Инженер-программист}}
  {IBM}
  {\EngRus{Moscow}{Москва}}
  {}
  {
    \begin{adjustwidth}{0.5cm}{0cm}
      \begin{itemize}
        \item \EngRus{Developed a Linux driver for SCSI devices for IBM z System.}
                     {Разрабатывал Linux драйвер для SCSI-устройств хранения данных.}
      \end{itemize}
    \end{adjustwidth}
  }

\cventry
  {09.2014 -- 08.2016}
  {\EngRus{Researcher}{Исследователь}}
  {\EngRus{Research Computing Center MSU}
          {Научно-исследовательский вычислительный центр МГУ}}
  {\EngRus{Moscow}{Москва}}
  {}
  {
    \begin{adjustwidth}{0.5cm}{0cm}
      \begin{itemize}
        \item \EngRus{Designed and developed an algorithm of repeats search in DNA.}{}
        \item \EngRus{Speeded it up using MPI and CUDA (almost linear scale on 4096 nodes).}{}
      \end{itemize}
    \end{adjustwidth}
  }

%-------------------------------------------------------------------------------
% TECHNICAL SKILLS
%-------------------------------------------------------------------------------

\section{\EngRus{Technical skills}{Технические навыки}}

\cvline
  {Languages}
  {Golang, Python (prior experience: Perl/Bash/C/Cpp)}
\cvline
  {CI/CD}
  {Linux, Docker, Kubernetes, GitLab CI}
\cvline
  {SQL}
  {PostgreSQL, Redis}

%-------------------------------------------------------------------------------
% EDUCATION
%-------------------------------------------------------------------------------

\section{\EngRus{Education}{Образование}}

\cventry
  {2010 -- 2015}
  {\EngRus{M.S. in Applied Mathematics and Computer Science}{Специалист по прикладной математике и информатике}}
  {\EngRus{CMC}{ВМК}}
  {\EngRus{Lomonosov Moscow State University}{МГУ им. М.В. Ломоносова}}
  {\EngRus{Moscow}{Москва}}
  {}

%-------------------------------------------------------------------------------
% PUBLICATIONS, AWARDS, OPEN SOURCE PROJECT, ADDITIONAL INFO AND FOOTER
%-------------------------------------------------------------------------------

% ---------- Publications ------------------------------------------------------
%\ifdetailed{
%  \renewcommand\refname{\EngRus{Publications}{Публикации}}
%  \begin{thebibliography}{0}
%
%  \bibitem{ISP}
%    \EngRus{{\itshape A.N. Pankratov, R.K. Tetuev, M.I. Pyatkov, V.P. Toigildin, N.N. Popova}
%            Spectral analytical method of recognition of inexact repeats in character sequences.~--~
%            Proceedings of the Institute for System Programming Volume 27 (Issue 6). 2015 y. pp. 335-344.
%            \href{http://www.ispras.ru/en/proceedings/isp_27_2015_6/isp_27_2015_6_335/}
%                 {\textcolor{blue}{\underline{Abstract}}}}
%           {{\itshape Панкратов А.Н., Тетуев Р.К., Пятков М.И., Тойгильдин В.П., Попова Н.Н.}
%            Спектрально-аналитический метод распознавания неточных повторов в символьных последовательностях.~--~
%            Труды Института системного программирования РАН Том 27. Выпуск 6. 2015 г.
%            \href{http://www.ispras.ru/proceedings/docs/2015/27/6/isp_27_2015_6_335.pdf}
%                 {\textcolor{blue}{\underline{Стр. 335-344}}}}
%
%    \bibitem{cuda}
%      \EngRus{{\itshape V.P. Toigildin} Research and development of parallel algorithm
%              for~genome blurred repeats search. -- CUDA Almanac, 2015 February.~--~
%              \href{http://www.nvidia.ru/docs/IO/141194/CUDA-\%D0\%B0\%D0\%BB\%D1\%8C\%D0\%BC\%D0\%B0\%D0\%BD\%D0\%B0\%D1\%85-feb-2015.pdf}
%                   {\textcolor{blue}{\underline{p.12}}}}
%             {{\itshape Тойгильдин В.П.} Разработка и исследование параллельного алгоритма
%              поиска неточных повторов в геноме. -- CUDA Альманах, Февраль 2015.~--~
%              \href{http://www.nvidia.ru/docs/IO/141194/CUDA-\%D0\%B0\%D0\%BB\%D1\%8C\%D0\%BC\%D0\%B0\%D0\%BD\%D0\%B0\%D1\%85-feb-2015.pdf}
%                   {\textcolor{blue}{\underline{Стр. 12}}}}
%
%  \end{thebibliography}
%}
% ---------- End Publications --------------------------------------------------

% ---------- Awards ------------------------------------------------------------
%\section{\EngRus{Awards}{Награды}}
%\cventry
%  {2014}
%  {\EngRus{CUDA Center of Excellence MSU Grant}
%          {Стипендия от CUDA Center of Excellence МГУ}%
%  }
%  {}
%  {\EngRus{Moscow}{Москва}}
%  {}
%  {\EngRus{\normalsize \href{http://ccoe.msu.ru/ru/node/64}{\textcolor{blue}{\underline{Won a grant}}}
%           for significant acceleration of computing for~my~research by~using GPU.}
%          {\normalsize \href{http://ccoe.msu.ru/ru/node/64}{\textcolor{blue}{\underline{Выиграл стипендию}}}
%          за существенное ускорение вычислений для своей исследовательской работы
%          за счет применения графических процессоров.}%
%  }
% ---------- End Awards --------------------------------------------------------

% ---------- Open Source -------------------------------------------------------
%\ifdetailed{
%  \section{\EngRus{Open Source Project}{Open Source Проекты}}
%  \cvline
%    {\href{https://github.com/vladisalv/mpisbars}{\textcolor{blue}{\underline{mpiSBARS}}}}
%    {\EngRus{Parallel program for recognition of extended inexact repeats in~the~genome.
%             MPI+CUDA model is used for better scalability on heterogeneous high performance systems.}
%            {Параллельная программа для поиска неточных протяженных повторов
%             в биологических последовательностях. Используется модель MPI+CUDA
%             для достижения лучшей масштабируемости на современных гетерогенных системах.}
%    }
%}
% ---------- End Open Source ---------------------------------------------------

% ---------- Additional information --------------------------------------------
%\ifdetailed{
%  \section{\EngRus{Additional information}{Дополнительная информация}}
%  \cvline
%    {\EngRus{Languages}{Языки}}
%    {\EngRus{English(intermediate), Russian(native)}
%            {Английский(средний уровень), Русский(родной)}}
%  \cvline
%    {\EngRus{Interests}{Интересы}}
%    {\EngRus{Comedy, Philosophy}{Комедия, философия}}
%}
% ---------- Additional information --------------------------------------------

% ---------- Footer ------------------------------------------------------------
\fancyfoot[C]{\scriptsize ~~~~~~~~~~~~~~~~~~~~~~~~~~~
              You can find recent and more detailed version cv
              \href{https://github.com/vladisalv/cv}{\underline{here}}.\newline
              Last updated: \today\-}
% ---------- End Footer --------------------------------------------------------

\end{document}
