\documentclass[12pt,a4paper,sans]{moderncv}
\usepackage[scale=0.75]{geometry}
\usepackage{fancyhdr}
\usepackage{changepage}
\graphicspath{{images/}} % path to images


%-------------------------------------------------------------------------------
%     create and set new toggle
%-------------------------------------------------------------------------------

%-----        create and set new toggle    -------------------------------------
\newtoggle{langEnglish}
\newtoggle{langRussian}

\toggletrue{langEnglish}
\togglefalse{langRussian}

\newtoggle{compact}
\newtoggle{detailed}

\toggletrue{compact}
\togglefalse{detailed}

%-------------------------------------------------------------------------------
% parse version of cv from \jobname
%-------------------------------------------------------------------------------

\newcommand\ifjobname[2]{
    \expandafter\ifstrequal\expandafter{\jobname}{cv_TOIGILDIN_VLADISLAV#1}
    {#2}
    {}
}

\moderncvtheme[blue]{classic} % theme by default
\ifjobname{_diff}{
    \input{style.tex}
}

\ifjobname{_eng_compact}{
    \toggletrue{langEnglish}
    \togglefalse{langRussian}
    \toggletrue{compact}
    \togglefalse{detailed}
}

\ifjobname{_rus_compact}{
    \togglefalse{langEnglish}
    \toggletrue{langRussian}
    \toggletrue{compact}
    \togglefalse{detailed}
}

\ifjobname{_eng_detailed}{
    \toggletrue{langEnglish}
    \togglefalse{langRussian}
    \togglefalse{compact}
    \toggletrue{detailed}
}

\ifjobname{_rus_detailed}{
    \togglefalse{langEnglish}
    \toggletrue{langRussian}
    \togglefalse{compact}
    \toggletrue{detailed}
}

%-------------------------------------------------------------------------------
%
%-------------------------------------------------------------------------------

\usepackage{polyglossia}   %% загружает пакет многоязыковой вёрстки
\iftoggle{langEnglish}{
    \setdefaultlanguage{english}  %% устанавливает главный язык документа
}{
    \setdefaultlanguage[spelling=modern]{russian}  %% устанавливает главный язык документа
    \setotherlanguage{english} %% объявляет второй язык документа
}

\iftoggle{compact}{
    \nopagenumbers{} % uncomment to suppress automatic page numbering for CVs longer than one page
}{
    \usepackage{lastpage}
    \pagestyle{fancy}
    \rfoot{\thepage/\pageref{LastPage}}
}

%-------------------------------------------------------------------------------
%
%-------------------------------------------------------------------------------

\defaultfontfeatures{Ligatures={TeX},Renderer=Basic}  %% свойства шрифтов по умолчанию
\setmainfont[Ligatures={TeX,Historic}]{FreeSans} %% задаёт основной шрифт документа
%\setsansfont{CMU Sans Serif}                    %% задаёт шрифт без засечек
%\setmonofont{CMU Typewriter Text}               %% задаёт моноширинный шрифт


\newcommand\EngRus[2]{\iftoggle{langEnglish}{#1}{\iftoggle{langRussian}{#2}{}}}

\newcommand\ifdetailed[1]{\iftoggle{detailed}{#1}{}}

\newcommand\ifcompact[1]{\iftoggle{compact}{#1}{}}

% \usepackage[usenames]{xcolor}  %%% this does not work = thanks to idiots from modernCV package.
%
\setlength{\hintscolumnwidth}{2.5cm}           % if you want to change the width of the column with the dates

%-------------------------------------------------------------------------------
% PERSONAL INFORMATION
%-------------------------------------------------------------------------------

\firstname{\EngRus{Vladislav}{Владислав}}
\familyname{\EngRus{Toigildin}{Тойгильдин}}
\title{Software Developer}
\address{\EngRus{Moscow, Russia}{Россия, Москва}}
\phone[mobile]{+7~916~576~66~39}
\email{vladisalv@yandex.ru}
\extrainfo{%
    \\%
    \includegraphics[height=14pt]{github_logo.png}~\httplink[vladisalv]{github.com/vladisalv}\\
    \includegraphics[height=14pt]{linkedin_logo.png}~\httplink[vladisalv]{linkedin.com/in/vladisalv}
}
\photo[64pt][0.4pt]{photo.jpg}
\renewcommand\refname{Selected publications}

%-------------------------------------------------------------------------------
\begin{document}
\makecvtitle

\newtoggle{DEBUG}
%\toggletrue{DEBUG}
\togglefalse{DEBUG}

\iftoggle{DEBUG}{
    \iftoggle{langEnglish}{You use english}{}
    \iftoggle{langRussian}{You use russian}{}
    \iftoggle{compact}{You use compact}{}
    \iftoggle{detailed}{You use detailed}{}
    \jobname
}

\section{\EngRus{Experience}{Опыт работы}}
\cventry{06.2015 -- 03.2016}
        {\EngRus{Software engineer}{Инженер-программист}}
        {IBM}
        {\EngRus{Moscow}{Москва}}
        {}
        {
          \ifcompact{
            \EngRus{Linux on z System SAN component development.}
                   {Разработка подсистемы SAN ОС Linux для аппаратной платформы IBM z System.}
          }
          \ifdetailed{
            \EngRus{Development of Linux driver (zfcp) for IBM z System (s390x) storage hardware.}
                   {Разработка Linux драйвера для устройств хранения данных аппаратной платформы IBM z System (s390x).}
            \footnotesize{
              \vfill
              \begin{adjustwidth}{0.5cm}{0cm}
                \EngRus{Worked in international team. Developed code for zfcp driver and internal perf tools. Did code review, performed test, developed and reviewed technical documentation.}
                       {Работал в интернациональной команде. Занимался разработкой zfcp драйвера и утилиты для измерений производительности (разработка исходного кода, документации, исследование производительности, ревью кода).}
                \EngRus{Main programming language: C}
                       {\newline Основной язык разработки: Си}
              \end{adjustwidth}
            }
          }
        }

\bigskip

\cventry{11.2013 -- 10.2014}
        {\EngRus{Technician}{Техник}}
        {\EngRus{Nuclear Safety Institute}{ИБРАЭ РАН}}
        {\EngRus{Moscow}{Москва}}
        {}
        {
          \EngRus{Development model of hydrodynamic process in liquids using CABARET scheme.}
                 {Разработка модели течения вязких жидкостей.}
          \ifdetailed{
            \footnotesize{
              \vfill
              \begin{adjustwidth}{0.5cm}{0cm}
                \EngRus{Designed and implemented simple GUI with Qt. Configured development enviroment. Trained team basic of *nix and feature of HPC software development. Designed and implemented few features in project code (Fortran).}
                       {Спроектировал и реализовал простенький пользовательский интерфейс, используя кроссплатформенную библиотеку Qt. Занимался настройкой и поддержкой среды разработки. Обучал команду основам работы в *nix системах и особенности разработки высокопроизводительных приложений.}
              \end{adjustwidth}
            }
          }
        }


\section{\EngRus{Education}{Образование}}
\cventry{2010 -- 2015}
        {\EngRus{Master in Applied Mathematics and Computer Science}{Специалист по прикладной математике и информатике}}
        {\EngRus{Lomonosov Moscow State University}{МГУ им. М.В. Ломоносова}}
        {\EngRus{Moscow}{Москва}}
        {\ifdetailed{\EngRus{Faculty of Computational Mathematics and Cybernetics}{Факультет Вычислительной математики и кибернетики}}}
        {
          \ifdetailed{
            \footnotesize{
            \vfill
            \EngRus{Qualification: specialist in mathematics and system programming}
                   {Квалификация: математик, системный программист}
            \newline
            \EngRus{Department of Supercomputers and Quantum Informatics}
                   {Кафедра Суперкомпьютеров и Квантовой Информатики}
            \newline
            \EngRus{Specialization: high performance computing}
                   {Специализация: высокопроизводительные вычисления}
            \newline
            \EngRus{Knowledges: Computer architecture and assembler language, Algorithms and Data structures, Parallel data processing, Operating systems, Databases, Mathematical analysis, Discrete mathematics, Numerical methods and others.\newline}{}
            \EngRus{Master dissertation "Research and development of parallel algorithm for genome blurred repeats search"}
                   {Дипломная работа "Разработка и исследование параллельного алгоритма поиска неточных повторов в геноме"}
            }
          }
        }


\section{\EngRus{Technical Skills}{Технические навыки}}
\cvline{Languages}{C, C++, Bash, Assembler, Perl(basic), Fortran(basic)}
\cvline{VCS}{Git}
\cvline{OS}{GNU/Linux, FreeBSD}
\cvline{HPC}{MPI, Cuda, OpenMP}
\cvline{Builder}{Make, Autotools}
\cvline{Others}{Qt(basic), Latex, Gnu plot}


\section{\EngRus{Additional information}{Дополнительная информация}}
\cvline{\EngRus{Languages}{Языки}}
       {\EngRus{English(pre-intermediate), Russian(native)}
               {Английский(продвинутое владение), Русский(родной)}}
\cvline{\EngRus{Interests}{Интересы}}{\EngRus{Improv theatre}{Сценическая импровизация}}


\fancyfoot[C]{\scriptsize ~~~~~~~~~~~~~~~~~~~~~~~~~~~ You can find recent and more detailed version cv \href{https://github.com/vladisalv/cv}{here}.\newline Last updated: \today\-}

%\section{Research Experience}
%\cventry{April 2009 --\\ February 2010 }{Senior Engineer}{Bullshit Institute of Technology}{\newline Study of Horseshit}{}{}
%
%\section{Fellowships and Awards}
%\cvline{2009}{Moscow Conference Award (Diploma of excellence)}
%
%\section{Professional Participation}
%\cvline{Membership}{Society of Bullshit.}
%
%\section{Computer skills}
%\cvitem{OS}{UNIX, Debian GNU/Linux, Windows NT.}
%
%\section{Fields of Scientific Interest}
%\cvitem{Image\,processing}{images deconvolution, noise analysis in CCD/CMOS-registered images.}
%
%% Publications
%\renewcommand\refname{Publications}
%\begin{thebibliography}{0}
%\bibitem{specmethod} Toi, jsl, 2016
%\end{thebibliography}
%
%\normalsize
%
%\section{Personal information}
%% \cventry{Date of birth}{June 20, 1982}{}{Russian Federation}{}{}
%\cventry{Languages}{English (IELTS, 7 Overall: 6.5L, 8.5R, 6.0W, 6.0S)}{Russian (native), Esperanto (basic)}{}{}{}
%


\end{document}
